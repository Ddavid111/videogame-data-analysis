\Chapter{Leíró statisztikák}

Ez a fejezet a korábbi fejezetekben előállított, egységesített \textit{merged} adathalmazra épül, amely a különböző forrásokból származó videojáték-metaadatok összefésült és normalizált nézetét tartalmazza. Az adat-előkészítési, összevonási és felbontási lépések eredményeként létrejött adatállomány lehetővé teszi a Steam \cite{steam} platformon megjelenő videojátékok jellemzőinek átfogó, statisztikai szemléletű vizsgálatát.

A fejezet célja az adatok alapvető szerkezeti és mennyiségi tulajdonságainak feltárása, valamint a legfontosabb mintázatok és tendenciák bemutatása leíró statisztikai eszközök segítségével. Az elemzések nem prediktív jellegűek, hanem az adathalmaz megértését, konzisztenciájának ellenőrzését és a későbbi gépi tanulási feladatok előkészítését szolgálják.

A fejezet első alfejezete az előzetes vizsgálatokra fókuszál, amelyek során az adatok alapvető eloszlásai, időbeli lefedettsége és főbb numerikus jellemzői kerülnek áttekintésre. Ezt követően a grafikonokra épülő alfejezet vizuális eszközök segítségével mutatja be a játékmegjelenések, értékelések és egyéb kulcsfontosságú attribútumok közötti összefüggéseket és trendeket.

\Section{Előzetes vizsgálatok}

Az előzetes vizsgálatok célja a \textit{merged} adathalmaz szerkezetének és
alapvető minőségi jellemzőinek áttekintése. Ebben a részben a források közötti
átfedések mértékét, a rekordok forrás szerinti megoszlását, valamint az összevonás
során releváns integritási problémákat ellenőrzöm, amelyek közvetlenül befolyásolják
a későbbi statisztikai elemzések és adatbázisba importálás megbízhatóságát.

\subsection{Források közötti átfedések -- Venn-diagram}

Az A, B és C forrásadatok közötti átfedések vizsgálatához Venn-diagram készült, amely az \textit{appid} azonosítók alapján szemlélteti, hogy mely rekordok érhetők el kizárólag egy adott adatforrásban, illetve melyek jelennek meg több forrás metszetében. A vizsgálat célja annak ellenőrzése volt, hogy az összevonás során mekkora a források közötti lefedettség, valamint milyen mértékű kiegészítő információ várható az egyes adathalmazoktól.

\begin{figure}[H]
    \centering
    \includegraphics[width=0.85\textwidth]{images/venn_diagram.png}
    \caption{Az A, B és C források közötti átfedések Venn-diagramja \textit{appid} alapján}
    \label{fig:venn_sources}
\end{figure}

A \ref{fig:venn_sources} ábrán látható, hogy a források közötti legnagyobb átfedés a B és C adathalmazok között figyelhető meg (\num{79888} közös \textit{appid}), míg az A adathalmaz jellemzően a B forrással fed át (\num{24433} közös \textit{appid}). Az egyedi rekordok száma a B és C források esetében alacsonyabb (\num{5731} csak B-ben, 161 csak C-ben), míg az A forrás különálló elemei (\num{1234}) és az A--C közös, de B-ben nem szereplő rekordok (\num{1400}) kisebb részarányt képviselnek. A diagram alapján megállapítható, hogy a D adathalmaz kialakítása során a B és C forrás együttes használata biztosítja a legnagyobb lefedettséget, míg az A adathalmaz elsősorban kiegészítő szerepet tölt be.

A Venn-diagram az \texttt{merge/venn\_diagram.png} állományba került mentésre, az elemszámokat tartalmazó összefoglaló táblázat pedig \texttt{venn\_table.csv} formátumban is elérhető.


\subsection{Forrásösszegzés (Source Summary)}

A merged master táblában minden rekordhoz tartozik egy \textit{sources} mező, amely jelzi, hogy az adott sor mely eredeti adatforrás(ok)ból származik (A, B, C). Ez lehet egyetlen forrás (például \textit{A}), vagy több forrás kombinációja (például \textit{B,C} vagy \textit{A,B,C}).

A forrásonkénti rekordösszesítő célja annak feltérképezése volt, hogy a végső D adathalmaz milyen arányban épül az egyes forrásokra, illetve milyen mértékű átfedés figyelhető meg közöttük. Az elemzés során meghatározásra kerül:
\begin{itemize}
    \item hány rekord származik kizárólag az A, B vagy C forrásból,
    \item hány rekord érkezik két forrás kombinációjából,
    \item valamint hány rekord található meg mindhárom adatforrásban.
\end{itemize}

Az összesítés eredménye egy CSV formátumú táblázatban kerül mentésre, amely a \texttt{source\_summary.csv} fájlban érhető el. Ez a táblázat jól áttekinthető módon mutatja meg az egyes forráskombinációkhoz tartozó rekordok számát, valamint azt is, hogy az adott kombináció tartalmazza-e az A, B vagy C forrást.

A forrásonkénti rekordösszesítés kulcsszerepet játszik annak megértésében, hogy az adathalmazok mennyire fedik egymást, és hogy az összevonás során a meghatározott prioritási sorrend (C → B → A) hogyan jelenik meg a végső adatmodellben.

Az alábbi \ref{lst:save} függvény mutatja be a forrásonkénti rekordösszesítő előállításának megvalósítását:

\begin{python}[caption={Forrásonkénti rekordösszesítő előállításának megvalósítása},captionpos=b,label={lst:save}]
def save_source_summary(d: pd.DataFrame, output_dir: str):

    summary = d["sources"].value_counts().reset_index()
    summary.columns = ["source_combination", "record_count"]

    summary["contain_A"] = summary["source_combination"].str.contains("A")
    summary["contain_B"] = summary["source_combination"].str.contains("B")
    summary["contain_C"] = summary["source_combination"].str.contains("C")

    output_file = os.path.join(output_dir, "source_summary.csv")
    summary.to_csv(output_file, index=False, encoding="utf-8-sig")

    logging.info(f"Source summary saved: {output_file}")
\end{python}

\subsection{Integritásvizsgálat}

Az integritásvizsgálat célja a merged tábla alapvető adatminőségi problémáinak feltárása, amelyek az összevonási folyamat során vagy a forrásadatok sajátosságai miatt jelenhetnek meg. Az ellenőrzések elsősorban a kulcsazonosítók, a kötelező mezők és a dátumformátumok konzisztenciájára fókuszálnak, mivel ezek hibái közvetlenül befolyásolják a későbbi normalizálási lépések és az SQL importálás biztonságát.

A vizsgálat az alábbi ellenőrzéseket hajtja végre:
\begin{itemize}
    \item duplikált \textit{appid} értékek azonosítása,
    \item hiányzó \textit{appid} értékek számlálása,
    \item hiányzó játéknevek (\textit{name}) ellenőrzése,
    \item hiányzó vagy üres forrásjelölés (\textit{sources}) vizsgálata,
    \item érvénytelen \textit{release\_date} értékek detektálása dátumkonverzióval.
\end{itemize}

Az ellenőrzés eredménye egy külön CSV jelentésben kerül mentésre \texttt{integrity\_report.csv} néven, amely a hibák számát ellenőrzésenként összesítve tartalmazza. Ez a jelentés az összefésülés-folyamat utólagos ellenőrzésére és dokumentálására is alkalmas.

Az alábbi \ref{lst:integrity} függvény szemlélteti az integritásvizsgálat megvalósítását:

\begin{python}[caption={Integritásvizsgálat megvalósítása},captionpos=b,label={lst:integrity}]
def validate_integrity(d: pd.DataFrame, output_dir: str):

    results = []

    results.append({
        "check": "Duplicate appids",
        "error_count": d["appid"].duplicated().sum()
    })

    results.append({
        "check": "Missing appids",
        "error_count": d["appid"].isna().sum()
    })

    if "name" in d.columns:
        results.append({
            "check": "Missing names",
            "error_count": d["name"].isna().sum()
        })

    if "sources" in d.columns:
        results.append({
            "check": "Missing sources",
            "error_count": (d["sources"] == "").sum()
        })

    if "release_date" in d.columns:
        invalid = pd.to_datetime(d["release_date"], 
        errors="coerce").isna().sum()
        results.append({
            "check": "Invalid release_date",
            "error_count": invalid
        })

    integrity_df = pd.DataFrame(results)

    output_file = os.path.join(output_dir, "integrity_report.csv")
    integrity_df.to_csv(output_file, index=False, encoding="utf-8-sig")

    logging.info(f"Integrity check completed, saved to {output_file}")
\end{python}


\Section{Grafikonok}

A fejezet következő része a legfontosabb leíró statisztikai eredményeket
vizuális formában mutatja be. A grafikonok célja, hogy szemléletesen kiemeljék
a megjelenések időbeli alakulását, a források szerinti különbségeket, valamint
a műfaji és szezonális mintázatokat. Az ábrák minden esetben a \textit{merged}
adathalmazból származó, egységesített adatok alapján készültek.

\subsection{Hisztogramok és trendvizualizációk}

\subsubsection{Játékmegjelenések 2010 előtt}

\begin{figure}[H]
    \centering
    \includegraphics[width=0.95\textwidth]{images/hist_pre2010.png}
    \caption{A Steam platformon megjelent játékok száma évente 2010 előtt}
    \label{fig:hist_pre2010}
\end{figure}

Ahogy azt a \ref{fig:hist_pre2010} ábrán is láthatjuk a 2010 előtti időszakban a megjelenésszámok viszonylag alacsonyak voltak, és jellemzően a hagyományos kiadói modell dominált. Ebben az érában a belépési küszöb magasabb volt, az önálló fejlesztők és kisebb stúdiók megjelenése még korlátozott szerepet játszott a platform kínálatában.

\subsubsection{Játékmegjelenések 2010 után}

\begin{figure}[H]
    \centering
    \includegraphics[width=0.95\textwidth]{images/hist_post2010.png}
    \caption{A Steam platformon megjelent játékok száma évente 2010 után}
    \label{fig:hist_post2010}
\end{figure}

A 2010 utáni időszakban a megjelenésszámok jelentős növekedése figyelhető meg a \ref{fig:hist_post2010} ábrán, különösen 2014-et követően. A 2020 és 2024 közötti években a megjelenések száma stabilan 10–20 ezer közötti tartományban mozgott évente, ami jól tükrözi a digitális disztribúció térnyerését és a platform nyitottabbá válását a kisebb fejlesztők számára.

\subsubsection{Összes játék megjelenése évente}

\begin{figure}[H]
    \centering
    \includegraphics[width=0.95\textwidth]{images/hist_all_years.png}
    \caption{A Steam \cite{steam} platformon megjelent játékok száma évente (1997--2025 májusáig)}
    \label{fig:hist_all_years}
\end{figure}

Az \ref{fig:hist_all_years} ábra jól szemlélteti a videojáték-megjelenések számának hosszú távú növekedését. A 2014 utáni időszakban különösen meredek emelkedés figyelhető meg, amely szoros összefüggésbe hozható a Steam Direct \cite{steamdirect} modell bevezetésével és az indie fejlesztők számának ugrásszerű növekedésével. A megjelenések száma 2024-ben érte el csúcspontját, több mint \num{21000} új címmel. A 2025-ös érték részleges adatokat tartalmaz, mivel az adathalmaz csak az adott évig rendelkezésre álló megjelenéseket tartalmazza.

\subsection{Forrás szerinti bontás (A, B, C kategória)}

Az összefésült adathalmaz lehetőséget ad arra, hogy a játékmegjelenéseket az eredeti
adatforrások szerint elkülönítve is vizsgáljuk. Ennek megfelelően az elemzés
három kategóriát különböztet meg:

\begin{itemize}
    \item \textbf{A dataset} – Steam \cite{steam} metaadatok korábbi, strukturált CSV-alapú forrása
    \item \textbf{B dataset} – SteamSpy \cite{steamspy} adatok (JSON-alapú, külső statisztikai forrás)
    \item \textbf{C dataset} – 2024--2025-ös, legfrissebb Steam adatgyűjtések
\end{itemize}

Az alábbi ábrák az egyes forrásokhoz tartozó játékmegjelenések számát mutatják
évenkénti bontásban.

\paragraph{A kategória}

\begin{figure}[H]
    \centering
    \includegraphics[width=0.9\textwidth]{images/hist_sources_A.png}
    \caption{Az A kategóriába tartozó játékmegjelenések évenkénti alakulása}
    \label{fig:hist_sources_A}
\end{figure}

Az \textbf{A} forrás esetében jól látható a \ref{fig:hist_sources_A} ábrán, hogy a megjelenések száma korlátozottabb, és elsősorban a korábbi évek játékait fedi le.
A 2015–2018 közötti időszakban átmeneti növekedés figyelhető meg,
amelyet azonban a forrás időbeli lefedettségének kifutása követ.
A 2018 utáni visszaesés nem piaci trendet, hanem az adatforrás
strukturális lezárulását tükrözi.

\paragraph{B kategória}

\begin{figure}[H]
    \centering
    \includegraphics[width=0.9\textwidth]{images/hist_sources_B.png}
    \caption{A B kategóriába tartozó játékmegjelenések évenkénti alakulása}
    \label{fig:hist_sources_B}
\end{figure}

A \textbf{B} forrás növekedése jóval dinamikusabb, ahogy azt a \ref{fig:hist_sources_B} ábrán is láthatjuk. A 2019 utáni időszakban különösen jelentős emelkedés figyelhető meg,
ami a SteamSpy \cite{steamspy} adatgyűjtési lefedettségének bővülésével magyarázható.
2025-ben már májusig több mint 2300 \textbf{B} kategóriás megjelenés történt,
azonban ez az érték részleges adatot reprezentál, mivel az év még nem zárult le.

\paragraph{C kategória}

\begin{figure}[H]
    \centering
    \includegraphics[width=0.9\textwidth]{images/hist_sources_C.png}
    \caption{A C kategóriába tartozó játékmegjelenések évenkénti alakulása}
    \label{fig:hist_sources_C}
\end{figure}

A \textbf{C} kategória tartalmazza a legfrissebb adatokat,
ezért itt figyelhető meg a legerőteljesebb növekedés.
A 2021 utáni időszakban a megjelenések száma meredeken emelkedik,
ami jól tükrözi a Steam platformon tapasztalható tartalombővülést
és az indie fejlesztések dominanciáját.
A 2025-ös érték ebben az esetben is részleges,
mivel az adathalmaz csak az év első hónapjait tartalmazza. A C forrás modern időszaki felfutását a \ref{fig:hist_sources_C} ábra mutatja be.


\subsection{Top 5 műfaj időbeli trendje}

Az alábbi \ref{fig:top5} ábra a Steam \cite{steam} platformon megjelent videojátékok közül az
öt leggyakoribb műfaj évenkénti megjelenésszámát mutatja be az adatbázis teljes időtartamára
(1997--2025 májusáig).

\begin{figure}[H]
    \centering
    \includegraphics[width=0.95\textwidth]{images/hist_genres_top5.png}
    \caption{A Top 5 műfaj időbeli trendje}
    \label{fig:top5}
\end{figure}

A legnépszerűbb műfajok a vizsgált időszakban az alábbiak voltak:
\begin{itemize}
    \item Indie
    \item Casual
    \item Action
    \item Adventure
    \item Simulation
\end{itemize}

Az \textbf{Indie} műfaj toronymagasan vezet, és a modern időszakban
(2014 után) jó közelítéssel a teljes PC-s játékkínálat bővülésének
indikátoraként értelmezhető. Az indie címek számának növekedése
szoros összefüggést mutat a Steam Direct \cite{steamdirect} modell bevezetésével,
amely jelentősen csökkentette a belépési küszöböt a fejlesztők számára.

A műfajok idősorai között rendkívül erős együttmozgás figyelhető meg.
A 2014–2024 közötti időszakban számított korrelációs együtthatók
minden műfajpár esetén kiemelkedően magas értékeket mutatnak.
Ez arra utal, hogy a különböző műfajok növekedése nem egymás rovására történik,
hanem egy közös, mögöttes piaci expanziót követ.

Hosszabb távon a volumenek aránya stabilnak tekinthető:
a \textit{Casual}, \textit{Action} és \textit{Adventure} műfajok
megjelenésszáma nagyságrendileg az \textit{Indie} megjelenések
50–70\%-a között alakul, míg a \textit{Simulation} műfaj részesedése
ennek körülbelül a fele. Ennek megfelelően a műfajok idősorai
nem tekinthetők függetlennek, és a teljes kínálat növekedése jól
közelíthető egy központi volumenindikátor (Indie) és közel
konstans műfajarányok segítségével.

\paragraph{Műfajok közötti korrelációk (2014--2024)}

A Top 5 műfaj idősorai közötti együttmozgást kvantitatív módon is
ellenőriztem. Az alábbi táblázatok a Pearson-féle lineáris korrelációt \cite{pearson},
illetve a Spearman-féle rangkorrelációt \cite{spearman} mutatják a 2014--2024 közötti
időszakra számítva. A korrelációs eredményeket a \ref{tab:pearson_top5} és
\ref{tab:spearman_top5} táblázatok foglalják össze.

\begin{table}[H]
\centering
\begin{tabular}{lccccc}
\toprule
 & Action & Adventure & Casual & Indie & Simulation \\
\midrule
Action      & 1.000 & 0.997 & 0.998 & 0.997 & 0.994 \\
Adventure   & 0.997 & 1.000 & 0.998 & 0.994 & 0.996 \\
Casual      & 0.998 & 0.998 & 1.000 & 0.997 & 0.996 \\
Indie       & 0.997 & 0.994 & 0.997 & 1.000 & 0.992 \\
Simulation  & 0.994 & 0.996 & 0.996 & 0.992 & 1.000 \\
\bottomrule
\end{tabular}
\caption{Pearson-korreláció a Top 5 műfaj idősorai között (2014--2024)}
\label{tab:pearson_top5}
\end{table}

\begin{table}[H]
\centering
\begin{tabular}{lccccc}
\toprule
 & Action & Adventure & Casual & Indie & Simulation \\
\midrule
Action      & 1.000 & 0.991 & 1.000 & 1.000 & 1.000 \\
Adventure   & 0.991 & 1.000 & 0.991 & 0.991 & 0.991 \\
Casual      & 1.000 & 0.991 & 1.000 & 1.000 & 1.000 \\
Indie       & 1.000 & 0.991 & 1.000 & 1.000 & 1.000 \\
Simulation  & 1.000 & 0.991 & 1.000 & 1.000 & 1.000 \\
\bottomrule
\end{tabular}
\caption{Spearman-korreláció a Top 5 műfaj idősorai között (2014--2024)}
\label{tab:spearman_top5}
\end{table}

A kétféle korrelációs mérőszám egybehangzóan rendkívül erős
együttmozgást jelez, ami alátámasztja, hogy a vizsgált műfajok
idősorai nem tekinthetők egymástól függetlennek.


\subsection{Éves növekedési faktor (logaritmikus skálán)}

Az alábbi \ref{fig:hist_growth_rates} ábra az egyes évek közötti növekedési faktort mutatja be,
vagyis azt, hogy az adott évben megjelent játékok száma
hányszorosára változott az előző évhez képest.
Az ábrázolás logaritmikus skálán történt, amely lehetővé teszi
a szélsőséges növekedési és visszaesési periódusok egyidejű
áttekintését.

\begin{figure}[H]
    \centering
    \includegraphics[width=0.95\textwidth]{images/hist_growth_rates.png}
    \caption{Éves növekedési faktor logaritmikus skálán}
    \label{fig:hist_growth_rates}
\end{figure}

A logaritmikus skála jól kiemeli a kiugró éveket, különösen
\textbf{2005} és \textbf{2014} környékén, amelyek a Steam \cite{steam} platform
strukturális változásaihoz köthetők.  
A 2014-es ugrás egyértelműen összefügg a Steam Direct \cite{steamdirect} modell
bevezetésével, amely jelentősen megnövelte az új megjelenések számát.

A növekedési faktor az egységérték (1) körüli tartományban
stagnáló piacot jelez, míg az ez alatti értékek visszaesést,
az efeletti értékek pedig expanziót mutatnak.
A 2015 utáni időszakban a növekedési faktor stabilizálódása figyelhető meg,
ami arra utal, hogy a platform elérte a folyamatos, de kevésbé
robbanásszerű bővülés szakaszát.

A \textbf{2025-ös érték} részleges adatokat tartalmaz,
mivel az adathalmaz csak az év első hónapjainak (májusig bezárólag)
megjelenéseit foglalja magában, ezért az adott év növekedési faktora
nem tekinthető teljes értékűnek.

\subsection{Év $\times$ nap hőtérkép}

Az alábbi \ref{fig:heatmap_year_day} hőtérkép az egyes évek és naptári napok mentén mutatja
a videojáték-megjelenések intenzitását.
A színezés a napi publikálások számát reprezentálja,
így egyszerre szemlélteti a hosszú távú növekedési trendet
és az esetleges szezonális mintázatokat.

\begin{figure}[H]
    \centering
    \includegraphics[width=0.95\textwidth]{images/heatmap_year_day.png}
    \caption{Év $\times$ nap hőtérkép a játékmegjelenések intenzitásáról}
    \label{fig:heatmap_year_day}
\end{figure}

A hőtérkép látványosan mutatja a publikálási intenzitás
hosszú távú növekedését és a szezonalitás jelenlétét.
A 2014 utáni időszakban a megjelenések sűrűsége folyamatosan emelkedik,
míg a 2023--2025 közötti években már szinte minden napra
több tucat új játék jut.

A mintázat alapján a publikálás nem egyenletes:
bizonyos időszakokban és napokon koncentráltabb aktivitás figyelhető meg.
Mivel azonban az év napjainak hétköznapra esése évről évre eltér,
a hét napjaihoz köthető hatások ebben az ábrázolásban
részben elmosódnak.

\subsection{Év $\times$ (ISO hét $\times$ hét napja) hőtérkép}

A \ref{fig:heatmap_year_weekday_aligned} ábra a videojáték-megjelenések
időbeli eloszlását szemlélteti egy olyan átrendezett időtengely mentén,
amely az egyes naptári napokat az ISO szabvány szerinti hét száma és
a hét napja alapján rendezi sorba.

Az ábra függőleges tengelye az éveket jelöli, míg a vízszintes tengely
az adott év összes napját tartalmazza úgy, hogy azok először az
ISO-hetek (1–53), azon belül pedig a hét napjai szerint követik egymást.
Ennek eredményeként a vízszintes tengelyen egymás mellett jelennek meg
azon napok, amelyek az év különböző időpontjaiban ugyanarra a
hétnapra estek.

A hőtérkép egy-egy cellája az adott évben, az adott ISO hét adott
napján megjelent játékok számát mutatja, ahol a színezés intenzitása
a publikálások számával arányos. A világosabb színek alacsonyabb,
a sötétebb színek magasabb megjelenésszámot jeleznek.

Az ilyen típusú igazítás előnye, hogy csökkenti a naptári eltolódások
hatását (például azt, hogy ugyanaz a dátum különböző években más
hétnapra esik), és lehetővé teszi a hét napjaihoz kötődő ciklikus
mintázatok azonosítását. A 2014 utáni időszakban jól látható,
hogy a megjelenések intenzitása nemcsak éves szinten növekszik,
hanem a hét napjai szerint is következetes szerkezetet mutat.

A modern években a magas intenzitású sávok szinte az év teljes hosszán
megjelennek, ami arra utal, hogy a publikálás nem kampányszerű,
hanem folyamatos jellegű, és stabil, strukturális piaci bővülést tükröz.

\section{Szezonális mintázatok – havi, heti és napi bontás}

A játékmegjelenések időbeli szerkezetét több idősíkban vizsgáltam,
mivel az eltérő aggregálási szintek különböző jellegű mintázatokat
emelnek ki. A havi, heti és napi bontás együttesen lehetővé teszi
a hosszú távú trendek és a rövidebb periódusú ciklikusság elkülönítését.

\subsection{Havi szezonális mintázatok}

\begin{figure}[H]
    \centering
    \includegraphics[width=0.95\textwidth]{images/seasonal_monthly_overlay.png}
    \caption{Játékmegjelenések havi bontásban – évek egymásra rajzolva}
    \label{fig:seasonal_monthly_overlay}
\end{figure}

A havi bontás jól szemlélteti (\ref{fig:seasonal_monthly_overlay} ábra) a hosszabb távú szezonalitást és az évről évre fokozódó volumen növekedést. A modern időszakban
(2014 után) a megjelenések száma gyakorlatilag minden hónapban
emelkedő trendet követ, jelentősebb, tartós visszaesések nélkül.
Ez arra utal, hogy a piac bővülése nem egy-egy kiemelt időszakhoz,
hanem az év egészéhez köthető.

\subsection{Heti szezonális mintázatok}

\begin{figure}[H]
    \centering
    \includegraphics[width=0.95\textwidth]{images/seasonal_weekly_overlay.png}
    \caption{Játékmegjelenések heti bontásban – évek egymásra rajzolva}
    \label{fig:seasonal_weekly_overlay}
\end{figure}

A heti bontás (\ref{fig:seasonal_weekly_overlay} ábra) kisimítja a napi szintű ingadozásokat, miközben az éven belül jelentkező periodikus mintázatok továbbra is jól
azonosíthatók maradnak. Az egyes évek görbéi
jelentős együttmozgást mutatnak, ami arra utal, hogy a publikálási
dinamika időben stabil szerkezetet követ. A csúcs- és alacsonyabb
aktivitású időszakok évről évre hasonló hetekhez köthetők.

\subsection{Napi mintázatok ISO-heti igazítással}

\begin{figure}[H]
    \centering
    \includegraphics[width=0.95\textwidth]{images/seasonal_weekly_overlay_iso.png}
    \caption{Heti bontás ISO-heti igazítással – a weekday-hatás csökkentése}
    \label{fig:seasonal_weekly_overlay_iso}
\end{figure}

A napi szintű adatok önmagukban rendkívül zajosak, elsősorban
a hét napjának hatása miatt. Ezért a napi megjelenéseket
ISO-heti bontásban aggregáltam, amellyel a weekday-hatás
jelentős része kiküszöbölhető.

Az így kapott idősorok (\ref{fig:seasonal_weekly_overlay_iso} ábra) már jól összevethetők egymással:
a csúcsok és visszaesések időben nagyrészt egybeesnek,
ami erős együttmozgásra és ismétlődő publikálási mintázatokra utal.
A 2020 utáni időszakban szinte minden héten tartósan magas
megjelenésszám figyelhető meg.

\section{2026-os előrejelzés logaritmikus skálán illesztett lineáris trend alapján}

\begin{figure}[H]
    \centering
    \includegraphics[width=0.95\textwidth]{images/forecast_2026_log_modern.png}
    \caption{Játékmegjelenések előrejelzése 2026-ra log-lineáris trend alapján}
    \label{fig:forecast_2026}
\end{figure}

A 2026-os becslést a \ref{fig:forecast_2026} ábra mutatja be.
A modern évek (2014–2024) adataira illesztett log-lineáris
(exponenciális) trend alapján a 2026-os évre
\textbf{megközelítőleg \num{35789} új játékmegjelenés} becsülhető.

A modell 2024-re visszatesztelve körülbelül \textbf{10\%-os túlbecslést}
mutatott, ezért a 2026-os érték \emph{optimista forgatókönyvként}
értelmezendő, és inkább felső becslésnek tekinthető. A növekedés üteme
a piac telítődése vagy platformszabályozási hatások miatt a következő
években mérséklődhet.

\subsection{Az előrejelzés pontosságának vizsgálata}

A 2026-os évre készített becslés megbízhatóságát
expanding-window típusú, egy-lépéses visszateszteléssel vizsgáltam.
A módszer lényege, hogy a modell minden lépésben kizárólag
az adott évnél korábbi adatokra illesztett logaritmikus trend alapján
becsüli a következő évet, majd az előrejelzést összeveti
a ténylegesen megfigyelt értékkel.

A vizsgálat során a tréninghalmaz kezdőéve rögzített (2014),
majd a tréningidőszak évről évre bővül.
A becslési hiba alakulását az abszolút százalékos hiba (APE)
segítségével értékeltem.

Az eredményeket a \ref{fig:expanding_backtest_log} ábra szemlélteti.

\begin{figure}[H]
    \centering
    \includegraphics[width=0.85\textwidth]{images/expanding_backtest_ape_log_modern.png}
    \caption{Expanding-window visszatesztelés – a becslési hiba alakulása a tréningévek számának függvényében}
    \label{fig:expanding_backtest_log}
\end{figure}

Az ábra alapján megfigyelhető, hogy kevés tréningadat esetén
a becslések jelentős hibával terheltek,
ami a növekedési trend instabil becslésére vezethető vissza.
Ahogy azonban a tréningévek száma növekszik,
a becslési hiba fokozatosan csökken és stabilizálódik.
A 10 éves tréningablaknál a relatív hiba már megközelítőleg 10\%,
ami a modell konzisztens viselkedését jelzi.

A teljes vizsgált időszakban a modell jellemzően felülbecsülte
a tényleges megjelenésszámot.
Ez arra utal, hogy a játékmegjelenések növekedése
nem tisztán logaritmikus jellegű,
és a piac bővülését strukturális tényezők
(például telítődési hatások) mérséklik.

\section{Modellvalidáció – 2024 visszateszt}

\begin{figure}[H]
    \centering
    \includegraphics[width=0.95\textwidth]{images/forecast_2024_log_modern.png}
    \caption{2024-es előrejelzés visszatesztje a log-lineáris modell alapján}
    \label{fig:forecast_2024}
\end{figure}

A 2024-es visszateszt eredményei a \ref{fig:forecast_2024} ábrán láthatók.
A modell teljesítményének ellenőrzésére a 2014–2023 közötti adatokon
tanított log-lineáris trend segítségével előrejeleztem a 2024-es évet.
A becslés \textbf{\num{23717}} megjelenést adott, míg a tényleges érték
\textbf{\num{21454}} volt, ami hozzávetőleg \textbf{10\%-os eltérést} jelent.

Ez az eredmény arra utal, hogy a modell a közelmúltban enyhén optimista,
ugyanakkor a növekedési trend irányát és nagyságrendjét helyesen ragadja meg,
így hosszabb távú extrapolációs alapként óvatosan alkalmazható.
