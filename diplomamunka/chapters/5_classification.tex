\Chapter{Osztályozási problémák}

Ebben a fejezetben a korábban előkészített és egységesített Steam\cite{steam}-adatokból
két, egymástól eltérő \textit{bináris osztályozási} feladatot fogalmazok meg és
vizsgálok. A két feladat közös jellemzője, hogy mindkettőnél a célváltozó két értéket vehet fel,
és a modellezés célja a játékok egyes tulajdonságai alapján a megfelelő osztály előrejelzése.
A módszertan mindkét esetben egységes: adatkijelölés $\rightarrow$ előfeldolgozás $\rightarrow$
tanító/teszt halmaz képzés $\rightarrow$ modellillesztés $\rightarrow$ teljesítménymérés.

A két vizsgált osztályozási probléma:
\begin{itemize}
    \item \textbf{Videójáték ``sikerességének'' becslése} (pozitív értékelési arány alapján képzett címke),
    \item \textbf{Multiplayer vs. singleplayer} kategória becslése (címkék és leírás alapján képzett címke).
\end{itemize}

A fejezet célja nem egyetlen konkrét algoritmus kiemelése,
hanem egy átlátható, reprodukálható osztályozási munkafolyamat bemutatása,
amely később új célváltozókra is kiterjeszthető.

\Section{Kutatási módszertan és feldolgozási lépések}

Mindkét osztályozási feladatnál az alábbi feldolgozási folyamatot alkalmaztam:

\begin{enumerate}
    \item \textbf{Cél (target) megfogalmazása:}
    a bináris osztálycímke definíciója (pl. sikeres/nem sikeres, multiplayer/nem multiplayer).
    \item \textbf{Adatok kijelölése:}
    azoknak a rekordoknak és attribútumoknak a kiválasztása, amelyek relevánsak a célváltozóhoz.
    \item \textbf{Előfeldolgozás és jellemzőképzés:}
    hiányzó értékek kezelése, numerikus változók skálázása, kategóriák binarizálása,
    illetve a szöveges mezők vektorizálása (TF--IDF\cite{tf-idf}).
    \item \textbf{Tanítás és tesztelés:}
    a tanító és teszt halmazok szétválasztása, majd modellillesztés.
    \item \textbf{Értékelés:}
    több metrika alapján (pontosság, precízió, recall, F1\cite{metrics}; ahol releváns, ROC--AUC\cite{roc-auc}),
    továbbá konfúziós mátrix\cite{metrics} és részletes osztályozási jelentés alapján.
\end{enumerate}

A módszertan előnye, hogy a két külön feladatra ugyanaz a \textit{logikai szerkezet} alkalmazható,
miközben a célváltozó és a felhasznált jellemzők a feladat sajátosságaihoz igazodnak.

\subsubsection{Alkalmazott teljesítménymérő metrikák}

Az osztályozási modellek teljesítményének értékelésére több különböző
metrikát alkalmaztam, amelyek a predikciók különböző aspektusait
ragadják meg.

Az alapvető mutatók közé tartozik a pontosság (accuracy),
a precízió (precision), a visszahívás (recall) és az F1-mérték\cite{metrics}.
Ezek a metrikák a konfúziós mátrix elemein alapulnak,
és információt adnak a modell tévesztéseinek jellegéről.

A modellek rangsorolási képességének értékelésére a
ROC--AUC (Receiver Operating Characteristic – Area Under Curve)\cite{roc-auc}
mutatót is alkalmaztam.
A ROC görbe a valódi pozitív arány (true positive rate)
és a hamis pozitív arány (false positive rate) kapcsolatát ábrázolja
különböző döntési küszöbértékek mellett.

A ROC--AUC érték a görbe alatti területet méri,
amely 0 és 1 közötti értéket vehet fel.
Az 1 érték tökéletes osztályozást,
a 0.5 érték pedig véletlenszerű besorolást jelent.
Ez a metrika küszöbfüggetlen módon értékeli a modell teljesítményét,
ezért különösen alkalmas kiegyensúlyozatlan osztályeloszlás esetén is.

A ROC--AUC mutató értelmezhető annak valószínűségeként,
hogy a modell egy véletlenszerűen kiválasztott pozitív példát
magasabb valószínűséggel sorol a pozitív osztályba,
mint egy véletlenszerűen kiválasztott negatív példát.
Ez különösen hasznos a videojátékok sikerességének becslése során,
ahol a cél nemcsak a bináris döntés,
hanem a játékok relatív rangsorolása is.



\Section{Vizsgálat 1: Videójáték sikerességének becslése}

\subsection{A probléma specifikációja (cél)}
A feladat célja annak előrejelzése, hogy egy játék \textit{sikeresnek} tekinthető-e.
A címke a felhasználói értékelésekből származtatott: a pozitív értékelések arányát
(\textit{positive ratio}) számítom, és küszöböléssel képezem a bináris osztályt.
Azoknál a rekordoknál, ahol nincs elegendő/értelmezhető értékelés (pl. 0 összes értékelés),
a címke nem értelmezhető, ezért ezek a rekordok a tanításhoz nem kerülnek felhasználásra.

\subsection{Adatkijelölés és jellemzők}
A sikeresség-becsléshez olyan változókat választottam, amelyek a piaci teljesítménnyel
és a játék jellemzőivel kapcsolatban állhatnak. A munkafüzetben az alábbi típusú jellemzők jelennek meg:

\begin{itemize}
    \item \textbf{Numerikus alapjellemzők:} például ár (\textit{price}), korhatár
    (\textit{required\_age}), megjelenési év (\textit{release\_date}), és további, adathalmazban elérhető számszerű mutatók.
    \item \textbf{Kategória jellegű változók:} műfajok és kategóriák bináris indikátorváltozókká alakítva
    (multi-hot\cite{multi-hot} reprezentáció), hogy a többértékű mezők is használhatók legyenek klasszikus modellekben.
\end{itemize}

A jellemzők kiválasztását megelőzően a numerikus változók közötti kapcsolatok
feltárására korrelációvizsgálatot végeztem annak ellenőrzésére, hogy az egyes
attribútumok nem hordoznak-e jelentős mértékben redundáns információt.
A vizsgálat eredményei alapján a numerikus változók között nem volt megfigyelhető
erős lineáris kapcsolat, ezért egyik változó kizárása sem volt indokolt.

A dimenziócsökkentési módszerek (például főkomponens-analízis) alkalmazhatóságát
szintén megvizsgáltam.
Az eredmények azt mutatták, hogy az információ nem koncentrálódik néhány
komponensbe, így a dimenziócsökkentés nem eredményezett volna lényegesen
egyszerűbb reprezentációt az értelmezhetőség romlása nélkül.
Ennek megfelelően a további elemzések során az eredeti jellemzők kerültek
felhasználásra.

A korrelációvizsgálat eredményeit a \ref{fig:corr_matrix} ábra,
a PCA kumulált magyarázott varianciáját pedig a \ref{fig:pca_variance} ábra szemlélteti
(a függelékben).

\begin{figure}[H]
    \centering
    \includegraphics[width=0.8\textwidth]{images/correlation_matrix.png}
    \caption{Numerikus változók korrelációs mátrixa}
    \label{fig:corr_matrix}
\end{figure}

\begin{figure}[H]
    \centering
    \includegraphics[width=0.7\textwidth]{images/pca_cumulative_variance.png}
    \caption{A numerikus változók PCA-alapú kumulált magyarázott varianciája}
    \label{fig:pca_variance}
\end{figure}

\subsection{Előfeldolgozás}
A kiválasztott jellemzők modellbarát formára alakításához
egységes előfeldolgozási lépéseket alkalmaztam.
A numerikus változókat skálázom,
a kategóriajellegű attribútumokat pedig bináris indikátorváltozókká alakítom.
Ennek célja, hogy a különböző típusú jellemzők egyetlen,
egységes bemeneti mátrixban jelenjenek meg, amely közvetlenül
felhasználható a tanuló algoritmusok számára.

\subsection{Tanítás/tesztelés és metrikák}
Az adathalmazt tanító és teszt halmazra bontottam 80--20\% arányban.
A felosztás során ügyeltem arra, hogy a sikeres és nem sikeres játékok aránya
mindkét részhalmazban megegyezzen az eredeti adathalmaz eloszlásával.
Ez biztosítja, hogy a modell tanítása és értékelése során egyik osztály se legyen
felül- vagy alulreprezentált.

A teljesítményt több metrika alapján értékeltem:

\begin{itemize}
    \item \textbf{Accuracy} (pontosság)\cite{metrics} – összes helyes besorolás aránya,
    \item \textbf{Precision} (precízió), \textbf{Recall, F1}\cite{metrics} – különösen hasznos, ha az osztályok eloszlása nem tökéletesen egyenletes,
    \item ahol releváns, \textbf{ROC--AUC}\cite{roc-auc} – rangsorolási minőség mérésére.
\end{itemize}

\subsection{Mennyiségi jellemzők és javasolt ábrák}

Az adathalmaz több olyan mennyiségi jellemzőt tartalmaz, amelyek számszerű formában írják le a videojátékok tulajdonságait és a felhasználói visszajelzéseket. 
Ezek közé tartozik többek között a játék ára, megjelenési éve, korhatár-besorolása, valamint a felhasználói értékelésekhez kapcsolódó mennyiségek.

A sikeresség mérésére használt pozitív értékelési arány folytonos változó, amelynek eloszlása önmagában is fontos információt hordoz. 
Az eloszlás vizsgálata rámutat arra, hogy a játékok többsége magas pozitív visszajelzési aránnyal rendelkezik, ugyanakkor kisebb részük alacsony értékelési arányt mutat. 
Ez az eloszlás indokolja a később alkalmazott bináris osztályozási küszöbérték bevezetését.
A pozitív értékelések arányának eloszlását a \ref{fig:positive_ratio_dist} ábra szemlélteti.

\begin{figure}[H]
    \centering
    \includegraphics[width=0.8\textwidth]{images/pos_rating_ratio_distribution.png}
    \caption{A pozitív értékelések arányának eloszlása a videojátékok között}
    \label{fig:positive_ratio_dist}
\end{figure}

Amennyiben a játékok megjelenési ideje is rendelkezésre áll, lehetőség nyílik a pozitív értékelések arányának időbeli vizsgálatára is. 
Az évek szerinti bontás alapján megfigyelhető, hogy a pozitív visszajelzések átlagos aránya nem állandó, hanem hosszabb távon változó tendenciát mutat. 
Ez arra utalhat, hogy a felhasználói értékelési szokások, illetve a platform jellege időben módosult.
A pozitív értékelések átlagos arányának alakulását az évek függvényében a \ref{fig:positive_ratio_trend} ábra mutatja be.

\begin{figure}[H]
    \centering
    \includegraphics[width=0.8\textwidth]{images/pos_rating_ratio_trend.png}
    \caption{A pozitív értékelések átlagos arányának alakulása az évek függvényében}
    \label{fig:positive_ratio_trend}
\end{figure}

A bemutatott ábrák nemcsak leíró statisztikai szerepet töltenek be, hanem megalapozzák a későbbi prediktív modellezést is, mivel rávilágítanak a sikerességi mutató eloszlására és időbeli viselkedésére.

\Section{Vizsgálat 2: Multiplayer vs. singleplayer osztályozás}

\subsection{A probléma specifikációja (cél)}
A második feladat célja annak meghatározása, hogy egy játék inkább
\textit{multiplayer} jellegű-e vagy \textit{singleplayer} jellegű.
A címkét a rendelkezésre álló kategória- és tagek alapján képeztem:
a multiplayerhez kapcsolódó jelölések (pl. \textit{Multi-player}, \textit{Online Multi-Player},
\textit{Co-op} stb.) jelenléte esetén a rekord a multiplayer osztályba kerül.

\subsection{Adatkijelölés és jellemzők}

Ehhez a feladathoz a játékok \textit{szöveges leírásait}
és a kapcsolódó metaadatokat használtam fel,
mivel a multiplayer jelleg elsősorban
a játékmechanikákra és funkciókra vonatkozó információkból
vezethető le.

A numerikus jellemzők (például ár, értékelésszám,
átlagos játékidő vagy Metacritic-pontszám)
előzetes vizsgálata során
korrelációanalízist és főkomponens-elemzést alkalmaztam.
Ezek az elemzések azt mutatták,
hogy a célváltozó (\textit{multiplayer vs. singleplayer})
csak gyenge kapcsolatot mutat a mennyiségi attribútumokkal,
valamint hogy a numerikus jellemzők
erősen redundáns információt hordoznak.
Ez alapján a numerikus változók
nem tekinthetők elsődleges prediktoroknak
a vizsgált osztályozási feladatban.

Ennek megfelelően a jellemzőképzés fókuszába
a szöveges és strukturált leíró információk kerültek,
amelyek közvetlenebb módon tükrözik
a játékok funkcionalitását és játékmenetét.

\begin{figure}[H]
    \centering
    \includegraphics[width=0.6\textwidth]{images/numeric_correlation_mp.png}
    \caption{A numerikus jellemzők és a célváltozó kapcsolata}
    \label{fig:numeric_correlation_mp}
\end{figure}

A numerikus jellemzők és a célváltozó kapcsolatát
a \ref{fig:numeric_correlation_mp} ábra szemlélteti.

\begin{figure}[H]
    \centering
    \includegraphics[width=0.6\textwidth]{images/pca_numeric_mp.png}
    \caption{A főkomponens-elemzés eredményei}
    \label{fig:pca_numeric_mp}
\end{figure}

A főkomponens-elemzés eredményeit
a \ref{fig:pca_numeric_mp} ábra mutatja.

A felhasznált jellemzők a következők voltak:
\begin{itemize}
    \item \textbf{Szöveges mezők:} a játék részletes leírása
    (\textit{about\_the\_game}) és rövid leírása
    (\textit{short\_description}),
    \item \textbf{Szöveges reprezentáció:}
    TF--IDF (Term Frequency – Inverse Document Frequency)\cite{tf-idf} vektorizálás,
    amely a gyakori, de kevésbé informatív szavakat leértékeli,
    míg a megkülönböztető kifejezéseket nagyobb súllyal emeli ki,
    \item \textbf{Strukturált jellemzők:}
    a játékokhoz tartozó tagek és műfajok,
    bináris (multi-hot)\cite{multi-hot} kódolás formájában.
\end{itemize}

Ez a jellemzőválasztás lehetővé tette,
hogy a modellek egyszerre használják ki
a természetes nyelvű leírásokban rejlő szemantikai információt
és a metaadatok explicit módon megadott kategorizálását.

\subsection{Előfeldolgozás}

A multiplayer vs. singleplayer osztályozási feladat során
a játékok szöveges leírásait numerikus reprezentációvá kellett alakítani,
hogy azok gépi tanulási modellek bemeneteként felhasználhatók legyenek.
Ennek érdekében a szöveges mezőket
TF--IDF (Term Frequency -- Inverse Document Frequency)\cite{tf-idf} vektorizálással
dolgoztam fel.

A TF--IDF módszer célja,
hogy az adott dokumentumban gyakran előforduló,
de a teljes szövegállományban ritkább szavakat nagyobb súllyal reprezentálja,
miközben a sok dokumentumban megjelenő,
kevésbé informatív kifejezések hatását csökkenti.
Ez különösen alkalmas olyan feladatokra,
ahol a megkülönböztető szavak és kifejezések
jelentős szerepet játszanak az osztályozásban.

A vektorizálás során
unigram és bigram kifejezések kerültek figyelembevételre,
valamint minimális dokumentumgyakorisági küszöb
(\textit{min\_df}) és maximális szókészletméret
(\textit{max\_features}) került beállításra
a ritka, zajos kifejezések kiszűrése érdekében.
Az előfeldolgozás részeként
a hiányzó leírásokat és az üres szövegeket eltávolítottam,
így biztosítva a konzisztens bemeneti adatkészletet.

A TF--IDF vektorizálás eredményeként
egy $(104\,701 \times 50\,000)$ dimenziójú,
ritka mátrix jött létre,
ahol a sorok az egyes játékok leírásait,
az oszlopok pedig a szókészlet elemeit
(unigramok és bigramok) reprezentálják.
A szókészlet mérete elérte az előre meghatározott
\textit{max\_features = 50\,000} korlátot,
ami a szövegállomány jelentős nyelvi változatosságára utal.

A nagy dimenziószám és a ritka reprezentáció
indokolttá teszi lineáris osztályozó algoritmusok alkalmazását,
mint a logisztikus regresszió\cite{logistic regression} vagy a lineáris SVM\cite{svm},
amelyek hatékonyan kezelik a magas dimenziós,
szórványos jellemzőtereket.

\subsection{Tanítás/tesztelés és metrikák}

A tanító/teszt felosztást stratifikált módon végeztem, hogy az osztálycímkék
eloszlása mindkét részhalmazban megőrződjön.
A modell teljesítményét az osztályozási feladatoknál szokásos metrikák
segítségével értékeltem (Accuracy, Precision, Recall, F1\cite{metrics}, valamint ahol
releváns, ROC--AUC\cite{roc-auc}), az előző alfejezetben bemutatott módon.

\subsection{Mennyiségi jellemzők és osztályeloszlás}

A multiplayer és singleplayer játékok arányát a
\ref{fig:mp_sp_distribution} ábra szemlélteti.
Az adathalmaz erősen kiegyensúlyozatlan, a singleplayer játékok
jelentős többségben vannak, ami indokolja az osztályozási modellek értékelésénél
a pontosság mellett további metrikák
(például precision, recall és F1-mérték)
alkalmazását.

\begin{figure}[H]
    \centering
    \includegraphics[width=0.6\textwidth]{images/eda_target_distribution_multiplayer.png}
    \caption{A célváltozó eloszlása: multiplayer vs singleplayer}
    \label{fig:mp_sp_distribution}
\end{figure}

A szöveges leírás hosszának eloszlását osztályonként
a \ref{fig:description_length_by_class} ábra szemlélteti.
Az ábra normalizált gyakoriságokat mutat,
a szélsőséges értékek vizuális hatásának csökkentése érdekében
a 99. percentilisig korlátozott tartományban.

Megfigyelhető, hogy mindkét osztály esetén az eloszlás
erősen jobbra ferde, hosszú jobb oldali farokkal rendelkezik,
ami kizárja a normalitás feltételezését.
Az eloszlás alakja lognormális jellegre utal,
ami összhangban van a szöveghossz multiplikatív természetével.

A multiplayer játékok leírásai jellemzően hosszabbak,
és eloszlásuk nagyobb szórást mutat a singleplayer játékokéhoz képest.
Ez arra utal, hogy a leírás hossza
megkülönböztető információt hordozhat az osztályok között,
ami indokolja a szöveges jellemzők bevonását
a későbbi modellezési lépések során.

\begin{figure}[H]
    \centering
    \includegraphics[width=0.75\textwidth]{images/eda_description_length_distribution_by_class_p99_normalized.png}
    \caption{A leírás hosszának eloszlása osztályonként (normalizált)}
    \label{fig:description_length_by_class}
\end{figure}

A log(1 + szószám) transzformáció után az eloszlások közel szimmetrikussá válnak
(lásd \ref{fig:description_length_by_class_log} ábra),
ami alátámasztja a lognormális jelleg feltételezését.

\begin{figure}[H]
    \centering
    \includegraphics[width=0.75\textwidth]{images/eda_description_length_distribution_by_class_log.png}
    \caption{A leírás hosszának eloszlása log(1 + szószám) transzformáció után}
    \label{fig:description_length_by_class_log}
\end{figure}

\Section{Összegzés és további vizsgálati lehetőségek}

Az 5. fejezetben két, egymástól eltérő bináris osztályozási problémát
fogalmaztam meg a Steam\cite{steam}-adathalmaz alapján.
A célváltozók definiálását részletes jellemzőelemzés és
eloszlásvizsgálat előzte meg,
amelyek rávilágítottak az egyes feladatok eltérő jellegére
és a releváns prediktorváltozók körére.

A sikeresség becslésére irányuló feladat esetében
a numerikus és kategóriajellegű változók együttes alkalmazása
indokoltnak bizonyult,
míg a multiplayer vs. singleplayer osztályozásnál
a szöveges leírások és strukturált metaadatok
hordozzák a legfontosabb megkülönböztető információt.
A leíráshossz eloszlásának vizsgálata azt is megmutatta,
hogy a szöveges jellemzők statisztikai tulajdonságai
nemlineáris jellegűek,
ami befolyásolja a később alkalmazott modellek megválasztását.

A bemutatott elemzések megalapozták a következő fejezetben tárgyalt
osztályozási modellek kiválasztását és értékelését.
További vizsgálati lehetőséget jelenthetne például
összetettebb szöveges reprezentációk
(beágyazások, neurális nyelvi modellek),
valamint alternatív küszöbértékek és költségérzékeny
osztályozási megközelítések elemzése.
