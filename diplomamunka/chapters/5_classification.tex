\Chapter{Osztályozási problémák}

Ebben a fejezetben a korábban előkészített és egységesített Steam\cite{steam}-adatokból
két, egymástól eltérő \textit{bináris osztályozási} feladatot fogalmazok meg és
vizsgálok. A két feladat közös jellemzője, hogy mindkettőnél a célváltozó két értéket vehet fel,
és a modellezés célja a játékok egyes tulajdonságai alapján a megfelelő osztály előrejelzése.
A módszertan mindkét esetben egységes: adatkijelölés $\rightarrow$ előfeldolgozás $\rightarrow$
tanító/teszt halmaz képzés $\rightarrow$ modellillesztés $\rightarrow$ teljesítménymérés.

A két vizsgált osztályozási probléma:
\begin{itemize}
    \item \textbf{Videójáték ``sikerességének'' becslése} (pozitív értékelési arány alapján képzett címke),
    \item \textbf{Multiplayer vs. singleplayer} kategória becslése (címkék és leírás alapján képzett címke).
\end{itemize}

A fejezet célja nem egyetlen konkrét algoritmus kiemelése,
hanem egy átlátható, reprodukálható osztályozási munkafolyamat bemutatása,
amely később új célváltozókra is kiterjeszthető.

\Section{Kutatási módszertan és feldolgozási lépések}

Mindkét osztályozási feladatnál az alábbi feldolgozási folyamatot alkalmaztam:

\begin{enumerate}
    \item \textbf{Cél (target) megfogalmazása:}
    a bináris osztálycímke definíciója (pl. sikeres/nem sikeres, multiplayer/nem multiplayer).
    \item \textbf{Adatok kijelölése:}
    azoknak a rekordoknak és attribútumoknak a kiválasztása, amelyek relevánsak a célváltozóhoz.
    \item \textbf{Előfeldolgozás és jellemzőképzés:}
    hiányzó értékek kezelése, numerikus változók skálázása, kategóriák binarizálása,
    illetve a szöveges mezők vektorizálása (TF--IDF\cite{tf-idf}).
    \item \textbf{Tanítás és tesztelés:}
    a tanító és teszt halmazok szétválasztása, majd modellillesztés.
    \item \textbf{Értékelés:}
    több metrika alapján (pontosság, precízió, recall, F1\cite{metrics}; ahol releváns, ROC--AUC\cite{roc-auc}),
    továbbá konfúziós mátrix\cite{metrics} és részletes osztályozási jelentés alapján.
\end{enumerate}

A módszertan előnye, hogy a két külön feladatra ugyanaz a \textit{logikai szerkezet} alkalmazható,
miközben a célváltozó és a felhasznált jellemzők a feladat sajátosságaihoz igazodnak.

\Section{Domain 1: Videójáték sikerességének becslése}

\subsection{A probléma specifikációja (cél)}
A feladat célja annak előrejelzése, hogy egy játék \textit{sikeresnek} tekinthető-e.
A címke a felhasználói értékelésekből származtatott: a pozitív értékelések arányát
(\textit{positive ratio}) számítom, és küszöböléssel képezem a bináris osztályt.
Azoknál a rekordoknál, ahol nincs elegendő/értelmezhető értékelés (pl. 0 összes értékelés),
a címke nem értelmezhető, ezért ezek a rekordok a tanításhoz nem kerülnek felhasználásra.

\subsection{Adatkijelölés és jellemzők}
A sikeresség-becsléshez olyan változókat választottam, amelyek a piaci teljesítménnyel
és a játék jellemzőivel kapcsolatban állhatnak. A munkafüzetben az alábbi típusú jellemzők jelennek meg:

\begin{itemize}
    \item \textbf{Numerikus alapjellemzők:} például ár (\textit{price}), korhatár
    (\textit{required\_age}), megjelenési év (\textit{release\_date}), és további, adathalmazban elérhető számszerű mutatók.
    \item \textbf{Kategória jellegű változók:} műfajok és kategóriák bináris indikátorváltozókká alakítva
    (multi-hot reprezentáció), hogy a többértékű mezők is használhatók legyenek klasszikus modellekben.
\end{itemize}

\subsection{Előfeldolgozás}
A pipeline-ban a numerikus változókat skálázom,
a kategóriákat pedig bináris oszlopokra bontom. A cél az, hogy a különböző típusú változók
egységes, modellbarát mátrixba kerüljenek.

\subsection{Tanítás/tesztelés és metrikák}
A tanító és teszt halmazokat stratifikált módon választottam szét,
hogy az osztályok aránya mindkét részben hasonló maradjon.
A teljesítményt több metrika alapján értékeltem:

\begin{itemize}
    \item \textbf{Accuracy} (pontosság)\cite{metrics} – összes helyes besorolás aránya,
    \item \textbf{Precision} (precízió), \textbf{Recall, F1}\cite{metrics} – különösen hasznos, ha az osztályok eloszlása nem tökéletesen egyenletes,
    \item \textbf{Konfúziós mátrix}\cite{metrics} – a tévesztések típusainak áttekintésére,
    \item ahol releváns, \textbf{ROC--AUC}\cite{roc-auc} – rangsorolási minőség mérésére.
\end{itemize}

\subsection{Mennyiségi jellemzők és javasolt ábrák}

Az adathalmaz több olyan mennyiségi jellemzőt tartalmaz, amelyek számszerű formában írják le a videojátékok tulajdonságait és a felhasználói visszajelzéseket. 
Ezek közé tartozik többek között a játék ára, megjelenési éve, korhatár-besorolása, valamint a felhasználói értékelésekhez kapcsolódó mennyiségek.

A sikeresség mérésére használt pozitív értékelési arány folytonos változó, amelynek eloszlása önmagában is fontos információt hordoz. 
Az eloszlás vizsgálata rámutat arra, hogy a játékok többsége magas pozitív visszajelzési aránnyal rendelkezik, ugyanakkor kisebb részük alacsony értékelési arányt mutat. 
Ez az eloszlás indokolja a később alkalmazott bináris osztályozási küszöbérték bevezetését.
A pozitív értékelések arányának eloszlását a \ref{fig:positive_ratio_dist} ábra szemlélteti.

\begin{figure}[H]
    \centering
    \includegraphics[width=0.8\textwidth]{images/pos_rating_ratio_distribution.png}
    \caption{A pozitív értékelések arányának eloszlása a videojátékok között}
    \label{fig:positive_ratio_dist}
\end{figure}

Amennyiben a játékok megjelenési ideje is rendelkezésre áll, lehetőség nyílik a pozitív értékelések arányának időbeli vizsgálatára is. 
Az évek szerinti bontás alapján megfigyelhető, hogy a pozitív visszajelzések átlagos aránya nem állandó, hanem hosszabb távon változó tendenciát mutat. 
Ez arra utalhat, hogy a felhasználói értékelési szokások, illetve a platform jellege időben módosult.
A pozitív értékelések átlagos arányának alakulását az évek függvényében a \ref{fig:positive_ratio_trend} ábra mutatja be.

\begin{figure}[H]
    \centering
    \includegraphics[width=0.8\textwidth]{images/pos_rating_ratio_trend.png}
    \caption{A pozitív értékelések átlagos arányának alakulása az évek függvényében}
    \label{fig:positive_ratio_trend}
\end{figure}

A bemutatott ábrák nemcsak leíró statisztikai szerepet töltenek be, hanem megalapozzák a későbbi prediktív modellezést is, mivel rávilágítanak a sikerességi mutató eloszlására és időbeli viselkedésére.

\Section{Domain 2: Multiplayer vs. singleplayer osztályozás}

\subsection{A probléma specifikációja (cél)}
A második feladat célja annak meghatározása, hogy egy játék inkább
\textit{multiplayer} jellegű-e vagy \textit{singleplayer} jellegű.
A címkét a rendelkezésre álló kategória- és tagek alapján képeztem:
a multiplayerhez kapcsolódó jelölések (pl. \textit{Multi-player}, \textit{Online Multi-Player},
\textit{Co-op} stb.) jelenléte esetén a rekord a multiplayer osztályba kerül.

\subsection{Adatkijelölés és jellemzők}
Ehhez a feladathoz a játékok \textit{szöveges leírását} és a kapcsolódó információkat használtam fel.
A jellemzőképzés alapja a szöveg vektorizálása:

\begin{itemize}
    \item \textbf{Szövegmezők:} például a játék leírása (\textit{about\_the\_game} vagy hasonló mezők),
    \item \textbf{Szöveges reprezentáció:} TF--IDF (Term Frequency – Inverse Document Frequency)\cite{tf-idf} vektorizálás, amely a gyakori, de kevésbé informatív szavakat
    leértékeli, és a megkülönböztető kifejezéseket kiemeli.
\end{itemize}

\subsection{Előfeldolgozás}

A multiplayer vs. singleplayer osztályozási feladat során
a játékok szöveges leírásait numerikus reprezentációvá kellett alakítani,
hogy azok gépi tanulási modellek bemeneteként felhasználhatók legyenek.
Ennek érdekében a szöveges mezőket
TF--IDF (Term Frequency -- Inverse Document Frequency)\cite{tf-idf} vektorizálással
dolgoztam fel.

A TF--IDF módszer célja,
hogy az adott dokumentumban gyakran előforduló,
de a teljes szövegállományban ritkább szavakat nagyobb súllyal reprezentálja,
miközben a sok dokumentumban megjelenő,
kevésbé informatív kifejezések hatását csökkenti.
Ez különösen alkalmas olyan feladatokra,
ahol a megkülönböztető szavak és kifejezések
jelentős szerepet játszanak az osztályozásban.

A vektorizálás során
unigram és bigram kifejezések kerültek figyelembevételre,
valamint minimális dokumentumgyakorisági küszöb
(\textit{min\_df}) és maximális szókészletméret
(\textit{max\_features}) került beállításra
a ritka, zajos kifejezések kiszűrése érdekében.
Az előfeldolgozás részeként
a hiányzó leírásokat és az üres szövegeket eltávolítottam,
így biztosítva a konzisztens bemeneti adatkészletet.

A TF--IDF vektorizálás eredményeként
egy $(104\,701 \times 50\,000)$ dimenziójú,
ritka mátrix jött létre,
ahol a sorok az egyes játékok leírásait,
az oszlopok pedig a szókészlet elemeit
(unigramok és bigramok) reprezentálják.
A szókészlet mérete elérte az előre meghatározott
\textit{max\_features = 50\,000} korlátot,
ami a szövegállomány jelentős nyelvi változatosságára utal.

A nagy dimenziószám és a ritka reprezentáció
indokolttá teszi lineáris osztályozó algoritmusok alkalmazását,
mint a logisztikus regresszió\cite{logistic regression} vagy a lineáris SVM\cite{svm},
amelyek hatékonyan kezelik a magas dimenziós,
szórványos jellemzőtereket.


\subsection{Tanítás/tesztelés és metrikák}
A tanító/teszt felosztást itt is stratifikált módon készítettem,
mivel a multiplayer/singleplayer címkék eloszlása torzított lehet.
A kiértékelés a klasszikus osztályozási metrikák mentén történt:

\begin{itemize}
    \item \textbf{Accuracy} (pontosság)\cite{metrics} – összes helyes besorolás aránya,
    \item \textbf{Precision} (precízió), \textbf{Recall, F1}\cite{metrics} – különösen hasznos, ha az osztályok eloszlása nem tökéletesen egyenletes,
    \item \textbf{Konfúziós mátrix}\cite{metrics} – a tévesztések típusainak áttekintésére,
    \item ahol releváns, \textbf{ROC--AUC}\cite{roc-auc} – rangsorolási minőség mérésére.
\end{itemize}

\subsection{Mennyiségi jellemzők és osztályeloszlás}

A multiplayer és singleplayer játékok arányát a
\ref{fig:mp_sp_distribution} ábra szemlélteti.
Az adathalmaz erősen kiegyensúlyozatlan, a singleplayer játékok
jelentős többségben vannak, ami indokolja az osztályozási modellek értékelésénél
a pontosság mellett további metrikák
(például precision, recall és F1-mérték)
alkalmazását.

\begin{figure}[H]
    \centering
    \includegraphics[width=0.6\textwidth]{images/eda_target_distribution_multiplayer.png}
    \caption{A célváltozó eloszlása: multiplayer vs singleplayer}
    \label{fig:mp_sp_distribution}
\end{figure}

A szöveges leírás hosszának eloszlását osztályonként
a \ref{fig:description_length_by_class} ábra szemlélteti.
Az ábra normalizált gyakoriságokat mutat,
a szélsőséges értékek hatásának csökkentése érdekében
a 99. percentilisig korlátozott tartományban.

Megfigyelhető, hogy a multiplayer játékok leírásai
jellemzően hosszabbak,
és eloszlásuk jobbra elnyúlóbb a singleplayer játékokéhoz képest.
Ez arra utal, hogy a leírás hossza
megkülönböztető információt hordozhat az osztályok között,
ami indokolja a szöveges jellemzők bevonását
a későbbi modellezési lépések során.

\begin{figure}[H]
    \centering
    \includegraphics[width=0.75\textwidth]{images/eda_description_length_distribution_by_class_p99_normalized.png}
    \caption{A leírás hosszának eloszlása osztályonként (normalizált)}
    \label{fig:description_length_by_class}
\end{figure}

\Section{Átvezetés a következő fejezetre}

A fejezetben bemutatott osztályozási problémákhoz kapcsolódó
adatelőkészítési lépések, jellemzőelemzések és célváltozó-definíciók
megalapozták a további modellezési feladatokat.
A bemutatott mennyiségi jellemzők és eloszlások alapján
mindkét vizsgált domain esetében indokolt a gépi tanulási
osztályozási módszerek alkalmazása.

A következő fejezet az előzőekben definiált osztályozási problémák
konkrét megoldására fókuszál.
Bemutatásra kerülnek az alkalmazott tanulóalgoritmusok,
a tanítás és validálás folyamata, valamint az egyes modellek
teljesítményének összehasonlítása a kiválasztott értékelési
metrikák segítségével.
