\Chapter{Adatok előkészítése}

Ebben a fejezetben a dolgozat során felhasznált adatok forrása és előkészítési folyamata kerül bemutatásra. A fejezet első része rövid áttekintést ad a Steam \cite{steam} szolgáltatásról, valamint az onnan származó, a Kaggle \cite{kaggle} platformon elérhető adathalmazok jellemzőiről, beleértve azok formátumát és mennyiségi tulajdonságait. Ezt követően a vizsgált adathalmazok részletesebb áttekintése történik meg, különös tekintettel az egyes adatkészletek külön notebookokban végzett elemzésére és az adatelőkészítés során szerzett gyakorlati tapasztalatokra.

\Section{Steam szolgáltatás és a kezelt adatai}

A következő alfejezetek röviden bemutatják a Steam \cite{steam} szolgáltatást mint
adatforrást, valamint a dolgozat során felhasznált adatok eredetét,
elérhetőségét és alapvető jellemzőit. Emellett áttekintésre kerülnek az
adatelőkészítés és elemzés során alkalmazott fejlesztőkörnyezetek és
szoftveres eszközök is.

\subsection{A Steam szolgáltatás áttekintése}

A Steam \cite{steam} egy digitális videojáték-disztribúciós platform, amelyet a Valve Corporation \cite{valve} üzemeltet, és amely világszerte az egyik legnagyobb online videojáték-áruháznak számít. A szolgáltatás lehetőséget biztosít videojátékok vásárlására, letöltésére és frissítésére, valamint különböző közösségi funkciók – például értékelések, fórumok és statisztikák – elérésére.

A Steam platformon elérhető videojátékokhoz számos, nyilvánosan hozzáférhető adat kapcsolódik. Ezek közé tartoznak többek között a játékok alapvető jellemzői (például cím, megjelenési dátum, fejlesztő, kiadó), műfaji besorolásuk, felhasználói értékelések, valamint egyes esetekben teljesítményre és népszerűségre vonatkozó mutatók. Az ilyen típusú adatok különösen alkalmasak statisztikai elemzésre és gépi tanulási módszerek alkalmazására.

A Steam által kezelt adatok mennyisége és sokfélesége miatt a platform gyakran szolgál adatforrásként kutatási és elemzési célokra. A játékokhoz kapcsolódó strukturált adatok lehetőséget biztosítanak különböző mintázatok feltárására, például a műfajok közötti különbségek, a felhasználói értékelések alakulása vagy a játékok népszerűségének vizsgálata során. Ennek köszönhetően a Steamhez kapcsolódó adathalmazok jól illeszkednek adatfeldolgozási és gépi tanulási feladatok bemutatásához. A Steam digitális videojáték-disztribúciós platform logóját a \ref{fig:steam_logo} ábra szemlélteti.

\begin{figure}[H]
    \centering
    \includegraphics[width=0.3\textwidth]{images/Steam_icon_logo.pdf}
    \caption{A Steam digitális videojáték-disztribúciós platform logója}\cite{steam_logo}
    \label{fig:steam_logo}
\end{figure}

\subsection{A Kaggle platform szerepe}

A Kaggle \cite{kaggle} egy online adatmegosztó és adatverseny-platform, amely elsősorban adatkutatási és gépi tanulási feladatokhoz biztosít nyilvánosan elérhető adathalmazokat. A platformon a felhasználók különböző forrásokból származó, dokumentált adatállományokat oszthatnak meg, amelyek kutatási és oktatási célokra egyaránt felhasználhatók.

A dolgozat során alkalmazott adathalmazok a Kaggle platformon kerültek publikálásra, és Steamhez kapcsolódó, nyilvános adatok feldolgozott formáit tartalmazzák. A Kaggle egységes hozzáférést, verziókezelést és leírást biztosít az adatkészletekhez, ami megkönnyíti azok összehasonlítható és reprodukálható feldolgozását. A Kaggle adatmegosztó és adatkutatási platform vizuális
azonosítóját a \ref{fig:kaggle_logo} ábra mutatja be.

\begin{figure}[H]
    \centering
    \includegraphics[width=0.25\textwidth]{images/Kaggle_logo.pdf}
    \caption{A Kaggle adatmegosztó és adatkutatási platform logója}\cite{kaggle_logo}
    \label{fig:kaggle_logo}
\end{figure}

\subsection{Az elemzés során használt eszközök}

Az adatok feldolgozása és elemzése Python \cite{python} programozási nyelv
felhasználásával történt. A fejlesztéshez a Visual Studio Code \cite{vscode}
fejlesztőkörnyezet, valamint Jupyter Notebook \cite{jupyter} alapú munkafüzetek
kerültek alkalmazásra, amelyek lehetővé tették az interaktív
adatelemzést és az egyes lépések dokumentálását.

Az adatok kezeléséhez és előfeldolgozásához elsősorban a \texttt{pandas} \cite{pandas}
és \texttt{NumPy} \cite{numpy} könyvtárak kerültek felhasználásra, míg a
vizualizációk elkészítéséhez a \texttt{matplotlib} \cite{matplotlib} és
\texttt{seaborn} \cite{seaborn} csomagok szolgáltak alapul. A gépi tanulási és
regressziós számítások során a \texttt{scikit-learn} \cite{scikit} könyvtár
egyes komponensei kerültek alkalmazásra.

\subsection{Letöltés, formátum és mennyiségi jellemzők}

A dolgozat során felhasznált adatok a Kaggle \cite{kaggle} platformon elérhető,
Steamhez \cite{steam} kapcsolódó nyilvános adathalmazokból származnak.
Az adatkészletek manuális letöltéssel kerültek beszerzésre a Kaggle webes felületén
keresztül, amelyhez felhasználói fiók megléte szükséges.
A letöltött állományok tömörített formában voltak elérhetők, és kicsomagolást
követően különálló adatfájlokat (CSV és JSON formátumú táblákat) tartalmaztak.

Az elsőként vizsgált adatkészlet a
\textit{Steam Store Games (Clean dataset)} \cite{a},
amely a továbbiakban \textit{A adathalmazként} kerül hivatkozásra.
Az adathalmaz a Steam áruházban elérhető videojátékok különböző jellemzőit
tartalmazza, többek között műfaji besorolásokat, leírásokat, címkéket és
becsült tulajdonosi adatokat.
Az adatok a Steam Store és a SteamSpy \cite{steamspy} API-k felhasználásával
kerültek összegyűjtésre, és elsősorban a 2019 májusáig megjelent játékokra
vonatkoznak.
Az A adathalmaz több, egymással logikailag összefüggő CSV fájlból áll,
amelyek külön relációként értelmezhetők; ezek rekord- és attribútumszáma
táblánként eltérő.

A második adathalmaz a Kaggle platformon
\textit{Steam Games Dataset} \cite{b} néven elérhető,
amely a továbbiakban \textit{B adathalmazként} kerül megjelölésre.
Az adathalmaz egyetlen, nagyméretű CSV/JSON alapú táblát tartalmaz,
amely a Steam API és a SteamSpy szolgáltatás adatain alapul.
A feltáró adatelemzés során a CSV formátum került alkalmazásra.
A B adathalmaz összesen \num{111452} rekordot és 40 attribútumot tartalmaz.

A harmadik vizsgált adatkészlet a
\textit{Steam Games Dataset 2025} \cite{c},
amely a továbbiakban \textit{C adathalmazként} kerül hivatkozásra.
Az adathalmaz a Steam áruház weboldalának automatizált feldolgozásával,
valamint a Steam API és a SteamSpy adatainak felhasználásával készült,
és 2025 márciusáig bezárólag tartalmaz adatokat.
Az adatok több CSV fájlban érhetők el, beleértve nyers és előfeldolgozott
változatokat is, amelyek táblánként eltérő attribútumkészlettel rendelkeznek.

Az adatkészletek eltérő szerkezete, fájlformátuma és attribútumkészlete
indokolttá tette az adatok előzetes tisztítását, normalizálását,
valamint egy egységesített adatszerkezet kialakítását,
amely a későbbi elemzési és gépi tanulási feladatok alapját képezi.
A felhasznált adathalmazok főbb mennyiségi jellemzőit,
táblánkénti bontásban, a \ref{tab:datasets} táblázat foglalja össze.

\begin{table}[H]
\centering
\begin{tabular}{|l|l|c|c|}
\hline
\textbf{Adathalmaz} & \textbf{Tábla} & \textbf{Rekordok száma} & \textbf{Attribútumok száma} \\
\hline
A & steam.csv & \num{27075} & 18 \\
A & steam\_description\_data.csv & \num{27334} & 4 \\
A & steam\_media\_data.csv & \num{27332} & 5 \\
A & steam\_requirements\_data.csv & \num{27319} & 6 \\
A & steamspy\_tag\_data.csv & \num{29022} & 372 \\
\hline
B & games.csv & \num{111452} & 40 \\
\hline
C & games\_march2025\_cleaned.csv & \num{89618} & 47 \\
C & games\_march2025\_full.csv & \num{94948} & 47 \\
C & games\_may2024\_cleaned.csv & \num{83646} & 46 \\
C & games\_may2024\_full.csv & \num{87806} & 46 \\
\hline
\end{tabular}
\caption{A felhasznált adathalmazok táblánkénti mennyiségi jellemzői}
\label{tab:datasets}
\end{table}

\section{A vizsgálandó adathalmaz áttekintése}

A vizsgált adathalmazok előzetes feltárása és elemzése különálló
Jupyter \cite{jupyter} munkafüzetekben történt, adathalmazonként
elkülönítve. Az A \cite{a}, B \cite{b} és C \cite{c} adathalmazok
önálló munkafüzetekben kerültek feldolgozásra, ami lehetővé tette az
egyes adatkészletek szerkezetének, tartalmának és sajátosságainak
részletes megismerését, valamint az adatokra jellemző problémák korai
azonosítását. Az egyes vizsgálatok részletes dokumentációja a dolgozathoz
mellékelt \textit{notebooks} jegyzékében érhető el.

A külön munkafüzetek alkalmazása elősegítette a feldolgozási folyamat
átláthatóságát, mivel az egyes adathalmazok eltérő szerkezettel,
formátummal és attribútumkészlettel rendelkeztek. Ez a megközelítés
lehetővé tette, hogy az adattisztítási, feltáró elemzési és
előfeldolgozási lépések az adott adathalmaz sajátosságaihoz igazodjanak,
anélkül hogy azok összekeveredtek volna más adatforrások feldolgozásával.

Az egyes adathalmazok esetében azonos alapvető vizsgálati lépések kerültek
elvégzésre. Ezek célja az adatok általános állapotának felmérése, valamint
a későbbi feldolgozási lépések megalapozása volt. A munkafüzetekben az
alábbi vizsgálatok történtek meg:
\begin{itemize}
    \item az adatok betöltése és szerkezetének áttekintése, beleértve az
    oszlopok számát, adattípusokat és a hiányzó értékek előfordulását,
    \item alapvető statisztikai leíró mutatók vizsgálata,
    \item a főbb attribútumok vizualizációja, például eloszlások,
    időbeli trendek és kategória-gyakoriságok megjelenítése,
    \item adatminőségi problémák, kiugró értékek és jellegzetes mintázatok
    azonosítása.
\end{itemize}

Az előzetes feltáró elemzés során néhány, az adathalmazok jellegét jól
szemléltető vizualizáció is készült. Ezek célja nem a részletes
statisztikai elemzés, hanem az adatok alapvető szerkezetének,
eloszlásainak és időbeli mintázatainak bemutatása.

Az \textit{A adathalmaz} esetében a játékok átlagos árának időbeli
alakulását vizsgáltam. A \ref{fig:mean} ábra az éves átlagárak változását
mutatja, amely alapján hosszabb távú trendek és árszintváltozások
azonosíthatók. Az ábra rávilágít arra, hogy az átlagár nem állandó,
hanem az egyes időszakokban számottevő ingadozást mutat, ami a
Steam-platform üzleti modelljének és a megjelenő játékok összetételének
változásával hozható összefüggésbe.

\begin{figure}[H]
    \centering
    \includegraphics[width=1.0\textwidth]{images/mean.png}
    \caption{Az A adathalmazban szereplő játékok átlagos ára évenként}
    \label{fig:mean}
\end{figure}

Szintén az \textit{A adathalmazhoz} kapcsolódóan a műfaji eloszlás
vizsgálata is megtörtént. A \ref{fig:top20} ábra a Top~20 műfaj
előfordulási gyakoriságát szemlélteti. Jól látható, hogy néhány műfaj
jelentős dominanciával rendelkezik, míg a ritkább kategóriák csak
korlátozott számban jelennek meg. Ez a jelenség fontos szempont az
adathalmaz későbbi elemzésekor, mivel a műfajok közötti egyenlőtlen
eloszlás torzíthat bizonyos statisztikai következtetéseket.

\begin{figure}[H]
    \centering
    \includegraphics[width=1.0\textwidth]{images/top20.png}
    \caption{Az A adathalmaz Top~20 műfajának előfordulási gyakorisága}
    \label{fig:top20}
\end{figure}

A \textit{B adathalmaz} esetében a játékok által támogatott platformok
szerinti megoszlás került vizsgálatra. A \ref{fig:platforms} ábra
bemutatja, hogy a Windows platform támogatása messze domináns, míg a
macOS és Linux rendszerek lényegesen kisebb arányban jelennek meg. Ez az
eltérés jól szemlélteti a platformfüggetlenség korlátait, valamint
magyarázatot adhat bizonyos platform-specifikus hiányosságokra az
adatokban.

\begin{figure}[H]
    \centering
    \includegraphics[width=1.0\textwidth]{images/platforms.png}
    \caption{A B adathalmazban szereplő játékok száma támogatott platformonként}
    \label{fig:platforms}
\end{figure}

A \textit{C adathalmaz} elemzése során a játékok Metacritic értékeinek
időbeli alakulása került előtérbe. A \ref{fig:metacritic} ábra a
Metacritic pontszámok és a kiadási év kapcsolatát ábrázolja. Az ábra
alapján megfigyelhető, hogy az értékelések jelentős szórást mutatnak
minden időszakban, ugyanakkor az idő előrehaladtával a pontszámok
eloszlása kiegyensúlyozottabbá válik. Ez arra utal, hogy a modern
időszakban a megjelenő játékok minősége stabilabb, illetve az
értékelési mechanizmus egységesebbé vált.

\begin{figure}[H]
    \centering
    \includegraphics[width=1.0\textwidth]{images/metacritic.png}
    \caption{A C adathalmazban szereplő játékok Metacritic értékei a kiadási év függvényében}
    \label{fig:metacritic}
\end{figure}

A feltáró vizsgálatok során szerzett tapasztalatok rávilágítottak arra,
hogy az adathalmazok közötti különbségek — például az eltérő
attribútumnevek, hiányzó értékek és formátumbeli eltérések — jelentős
hatással vannak a későbbi összevonási és normalizálási lépésekre. A
kezdeti, elkülönített feldolgozás ezért kulcsszerepet játszott egy
egységes, közös adatszerkezet kialakításának előkészítésében. Az itt
bemutatott előzetes elemzések eredményei szolgáltak alapul a következő
fejezetben részletezett adattisztítási, normalizálási és
adathalmaz-összevonási folyamatokhoz.
