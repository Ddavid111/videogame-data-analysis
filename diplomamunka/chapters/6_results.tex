\Chapter{Osztályozási problémák és megoldásuk}

Ebben a fejezetben a korábban definiált osztályozási problémák
konkrét megoldásának bemutatása és értelmezése történik.
A cél nem csupán a prediktív teljesítmény számszerű értékelése,
hanem annak vizsgálata is, hogy az alkalmazott modellek
milyen jellemzők alapján hozzák meg döntéseiket,
és ezek a döntések mennyiben tekinthetők értelmezhetőnek.

A fejezet két, eltérő jellegű problématerületre (domainre) bontva
mutatja be az osztályozási feladatokat.
Az első domain a videojátékok sikerességének becslésére fókuszál
felhasználói visszajelzések és strukturált metaadatok alapján,
míg a második domain a játékok multiplayer és singleplayer jellegének
automatikus felismerését vizsgálja szöveges leírások
és kapcsolódó metaadatok felhasználásával.

Mindkét esetben lineáris osztályozási módszerek kerültek alkalmazásra,
amelyek megfelelő egyensúlyt biztosítanak
a teljesítmény, a számítási hatékonyság
és a modellértelmezhetőség között.
A fejezetben bemutatott eredmények
megalapozzák a későbbi összehasonlító elemzést,
amely a különböző módszerek és jellemzők hatását vizsgálja.

\Section{Domain 1: Videójáték sikerességének elemzése}

\subsection{A sikeresség-becslés eredményeinek értelmezése}

A videojátékok sikerességének becslésére bináris osztályozási feladat került megfogalmazásra,
ahol a cél annak eldöntése volt, hogy egy adott játék a felhasználói értékelések alapján
sikeresnek tekinthető-e.
A modell teljesítményének értelmezéséhez nemcsak az összesített metrikák,
hanem a konfúziós mátrix is fontos információt szolgáltat.

A logisztikus regresszió\cite{logistic regression} modell eredményeit a \ref{fig:confusion_success} ábra szemlélteti.
A konfúziós mátrix alapján megfigyelhető, hogy a modell a sikeres játékokat
nagyobb arányban ismeri fel helyesen, mint a nem sikereseket.
Ez összhangban van azzal, hogy az adathalmazban a sikeres játékok aránya enyhén domináns.

\begin{figure}[H]
    \centering
    \includegraphics[width=0.6\textwidth]{images/confusion_matrix_logistic_regression.png}
    \caption{Konfúziós mátrix a videojátékok sikerességének becslésére (logisztikus regresszió)}
    \label{fig:confusion_success}
\end{figure}

A mátrixból látható, hogy a tévesztések jelentős része a hamis pozitív és hamis negatív
besorolásokból adódik.
A modell viselkedése alapján megállapítható, hogy inkább optimista becslést ad:
gyakrabban sorol nem sikeres játékot a sikeres kategóriába, mint fordítva.
Ez a tulajdonság bizonyos alkalmazási esetekben előnyös lehet, amennyiben a cél
a potenciálisan sikeres címek minél nagyobb arányú azonosítása.

A modell teljesítményét jellemző ROC--AUC\cite{roc-auc} érték azt mutatja,
hogy a becslés nem csupán bináris döntésként értelmezhető,
hanem a játékok sikeresség szerinti rangsorolására is alkalmas.
Ez azt jelenti, hogy a modell általában magasabb siker-valószínűséget rendel
a valóban sikeres játékokhoz, mint a nem sikeresekhez,
ami megerősíti a modell használhatóságát elemzési célokra.

\subsection{A sikerességet befolyásoló jellemzők értelmezése}

A prediktív teljesítmény értékelése mellett fontos szempont annak vizsgálata is,
hogy a modell mely bemeneti változókat tekinti meghatározónak a videojátékok
sikerességének becslése során.
Ennek érdekében a logisztikus regresszió modell koefficienseit elemeztem,
mivel ezek közvetlenül értelmezhető információt adnak a változók hatásának
irányáról és relatív erősségéről.

A legnagyobb abszolút értékű koefficiensekkel rendelkező változókat a
\ref{fig:feature_importance_logreg} ábra mutatja be.
Pozitív koefficiens esetén az adott jellemző növeli a sikeres besorolás
valószínűségét, míg negatív érték csökkentő hatásra utal.

\begin{figure}[H]
    \centering
    \includegraphics[width=0.8\textwidth]{images/logistic_regression_feature_importance.png}
    \caption{A legfontosabb változók a logisztikus regresszió modell alapján}
    \label{fig:feature_importance_logreg}
\end{figure}

Az eredmények alapján megfigyelhető, hogy a sikerességhez elsősorban nem
technikai vagy árazási tényezők járulnak hozzá, hanem a játék (illetve szoftver)
jellegére és használati módjára utaló attribútumok.
Pozitív hatást mutat többek között az egyjátékos mód jelenléte,
valamint a közösségi és kényelmi funkciókhoz kapcsolódó jellemzők
(például Family Sharing, Steam Cloud vagy Steam Workshop).

Ezzel szemben negatív irányú hatás figyelhető meg bizonyos üzleti modellek,
például az ingyenes (Free to Play) játékok esetében, illetve egyes speciális
szoftverkategóriákhoz kapcsolódó változóknál.
Ez arra utalhat, hogy ezeknél a címeknél a felhasználói elégedettség
és az értékelési szokások eltérő mintázatot követnek.

Az elemzés rávilágít arra, hogy a logisztikus regresszió nem pusztán
feketedobozos előrejelzőként használható, hanem lehetőséget ad a modell
döntéseinek kvalitatív értelmezésére is.

\subsection{A jellemzők számának hatása a prediktív teljesítményre}

A jellemzők fontossági sorrendje alapján megvizsgáltam,
hogyan változik a modell teljesítménye a felhasznált jellemzők számának csökkentésével.

Az eredmények azt mutatják, hogy már kis számú jellemző esetén is viszonylag jó
prediktív teljesítmény érhető el.
Például az öt legfontosabb jellemző használatával a modell pontossága 62.8\%,
míg az összes jellemző felhasználásával 65.3\%-ot ér el.

Ez arra utal, hogy a sikeresség előrejelzéséhez szükséges információ jelentős része
néhány kulcsjellemzőben koncentrálódik,
és a további jellemzők csak mérsékelt mértékben javítják a prediktív teljesítményt.

A jellemzők számának és a modell teljesítményének kapcsolatát a
\ref{fig:feature_count_vs_performance} ábra szemlélteti.

\begin{figure}[H]
\centering
\includegraphics[width=0.8\textwidth]{images/feature_count_vs_performance.png}
\caption{A jellemzők számának hatása a modell teljesítményére}
\label{fig:feature_count_vs_performance}
\end{figure}

\Section{Domain 2: Multiplayer vs. singleplayer osztályozás}

A második vizsgált osztályozási probléma a videojátékok multiplayer és
singleplayer kategóriákba sorolására irányult.
A bináris osztályozási feladat célja annak meghatározása volt,
hogy egy adott játék rendelkezik-e többjátékos funkcionalitással,
a játékok szöveges leírásai és strukturált metaadatai alapján.

A feladat megoldására több különböző modellt is kipróbáltam,
amelyek eltérő jellemzőkészletekre épültek.
A szöveges leírások feldolgozásához TF--IDF\cite{tf-idf} alapú
vektorizálást alkalmaztam, amely a szavak relatív fontosságát
számszerűsíti a dokumentumok között, és hatékony bemeneti
reprezentációt biztosít lineáris osztályozási modellek számára.
Alternatív megközelítésként lehetőség lett volna fejlettebb
természetes nyelvfeldolgozó eszközök, például a spaCy\cite{spacy}
könyvtár alkalmazására is, amely támogatja a tokenizálást,
lemmatizálást és egyéb nyelvi elemzési lépéseket.
A jelen vizsgálatban azonban a TF--IDF reprezentáció önmagában is
megfelelő teljesítményt biztosított a feladat megoldásához.

Az eredmények alapján a legjobb teljesítményt a
TF--IDF alapú szöveges jellemzőket és a strukturált metaadatokat
együttesen alkalmazó lineáris SVM (LinearSVC)\cite{svm} modell nyújtotta.
Ez a megoldás nemcsak az összesített pontosság,
hanem a kiegyensúlyozottabb osztályonkénti teljesítmény tekintetében is
kedvezőbbnek bizonyult.

A modell eredményeit a \ref{fig:confusion_multiplayer} ábra
konfúziós mátrixa szemlélteti.

\begin{figure}[H]
    \centering
    \includegraphics[width=0.6\textwidth]{images/confusion_matrix_tfidf_linearsvc_combined.png}
    \caption{Konfúziós mátrix a multiplayer vs. singleplayer osztályozási feladatra (Combined LinearSVC)}
    \label{fig:confusion_multiplayer}
\end{figure}

A konfúziós mátrix alapján megfigyelhető,
hogy a modell mindkét osztályt magas pontossággal képes elkülöníteni,
miközben a kisebb elemszámú multiplayer osztály esetében is
megfelelő visszahívási arányt ér el.
Ez különösen fontos az adathalmaz kiegyensúlyozatlansága miatt,
mivel a singleplayer játékok jelentős többségben vannak.
Az eredmények azt mutatják,
hogy a strukturált jellemzőkkel kiegészített szöveges reprezentáció
hatékonyan segíti a többjátékos mechanikák felismerését.

\subsection{A multiplayer besorolást befolyásoló jellemzők értelmezése}

A prediktív teljesítmény vizsgálata mellett
kiemelt figyelmet fordítottam a modell döntéseinek értelmezhetőségére is.
A lineáris SVM modell tanult súlyainak elemzése lehetőséget biztosított annak
feltárására, hogy mely jellemzők járulnak hozzá legnagyobb mértékben
a multiplayer, illetve a singleplayer besoroláshoz.

Az elemzés alapján megállapítható,
hogy a multiplayer irányba elsősorban olyan szöveges kifejezések
és strukturált tagek tolódnak el,
amelyek explicit módon többjátékos mechanikákra utalnak.
Ilyenek például a \textit{multiplayer}, \textit{online},
\textit{co-op}, \textit{pvp} vagy \textit{split screen} kifejezések,
amelyek mind a játékok leírásaiban,
mind a metaadatokban következetesen megjelennek.

Ezzel szemben a singleplayer besorolást inkább narratív,
tanulási vagy egyéni játékélményhez kötődő kifejezések támogatják,
mint például a \textit{single player}, \textit{prologue},
\textit{tutorial} vagy \textit{novel}.
Ez arra utal, hogy a modell döntései nem véletlenszerű korrelációkon,
hanem a játékmechanikákhoz és felhasználói élményhez
szorosan kapcsolódó szemantikai mintázatokon alapulnak.

Az eredmények összességében azt mutatják,
hogy a szöveges leírások és a strukturált jellemzők kombinációja
nemcsak a prediktív teljesítményt javítja,
hanem a modell döntéseinek értelmezhetőségét is megőrzi,
ami fontos szempont az elemzési és kutatási célú alkalmazások esetében.
