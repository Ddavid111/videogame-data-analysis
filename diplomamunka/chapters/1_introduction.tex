\Chapter{Bevezetés}

A videójátékok területén keletkező adatok elemzése napjainkban egyre fontosabb szerepet tölt be mind a kutatás, mind az ipari alkalmazások szempontjából. A Steam \cite{steam} platformon elérhető videójátékok jellemzőire, játékmenetére és teljesítményére vonatkozó adatok megfelelő feldolgozása lehetőséget teremt különböző mintázatok felismerésére, valamint prediktív és osztályozási problémák megoldására. A nagy mennyiségű és heterogén adatok kezelése azonban megfelelő adat-előkészítési és elemzési módszereket igényel.

A diplomamunka során három, a Kaggle \cite{kaggle} platformon elérhető, Steamhez kapcsolódó videojáték-adathalmaz kerül feldolgozásra. Az adatkészletek kezdetben különálló munkafüzetekben kerülnek elemzésre, az egyes adathalmazok szerkezetének és minőségének megismerése érdekében. Az előzetes vizsgálatot követően az adatok tisztítása és normalizálása történik meg annak érdekében, hogy az eltérő forrásból származó adatok egységes formában legyenek kezelhetők.

A normalizált adathalmazok ezt követően összevonásra kerülnek egy egységes, nagyobb adatkészletté. Az adatbázis logikus felépítése és a hatékony elemzés érdekében az összevont adathalmaz további, tematikusan elkülönülő táblákra kerül bontásra, biztosítva az adatok átlátható kezelését.

Az előkészített adatokon statisztikai és feltáró vizsgálatok kerülnek elvégzésre az adatok közötti összefüggések feltárása érdekében. Ezt követően a dolgozat a gépi tanulás területére fókuszál, különös tekintettel az osztályozási problémák megoldására. A Python \cite{python} programozási nyelv és annak gépi tanulást támogató könyvtárai segítségével különböző osztályozó algoritmusok kerülnek alkalmazásra és összehasonlításra.

A dolgozat célja annak bemutatása, hogy a megfelelő adat-előkészítés és -feldolgozás milyen mértékben befolyásolja a gépi tanulási modellek teljesítményét, valamint hogy az alkalmazott osztályozási módszerek mennyire hatékonyan alkalmazhatók videojáték-adatok elemzésében.
