\Chapter{Módszerek/eredmények összehasonlítása}

Ebben a fejezetben az előző fejezetekben bemutatott osztályozási feladatokra
alkalmazott különböző gépi tanulási módszerek teljesítményét hasonlítom össze.
A cél annak vizsgálata, hogy az eltérő modellek és jellemzőkészletek
milyen mértékben befolyásolják a prediktív teljesítményt,
különös tekintettel az osztályok közötti kiegyensúlyozottságra
és a kisebbségi osztály felismerésére.

\Section{Domain 1: Videójáték sikerességének elemzése}

\subsection{Alternatív osztályozási módszerek összehasonlítása}

A logisztikus regresszió\cite{logistic regression} mellett több további, gyakran alkalmazott osztályozási módszert is
megvizsgáltam a videojátékok sikerességének becslésére.
Mivel a jellemzőmátrix és a célváltozó már rendelkezésre állt, az összehasonlítás egységes
adatfelosztás mellett, alapértelmezett paraméterezéssel történt.

Az összehasonlítás során Random Forest\cite{random forest} és Gradient Boosting\cite{gradient boosting} modellek kerültek kipróbálásra.
A modellek teljesítményét a ROC--AUC\cite{roc-auc} mutató alapján értékeltem, mivel ez a metrika
küszöbfüggetlen módon méri a modellek rangsorolási képességét.
Az eredményeket a \ref{tab:model_comparison} táblázat foglalja össze.

\begin{table}[h]
    \centering
    \begin{tabular}{l c}
        \hline
        \textbf{Modell} & \textbf{ROC--AUC} \\
        \hline
        Logisztikus regresszió & 0.717 \\
        Random Forest & 0.744 \\
        Gradient Boosting & 0.754 \\
        \hline
    \end{tabular}
    \caption{Különböző osztályozási modellek összehasonlítása ROC--AUC alapján}
    \label{tab:model_comparison}
\end{table}

Az eredmények alapján megfigyelhető, hogy a nemlineáris modellek
(Random Forest és Gradient Boosting) magasabb ROC--AUC értéket értek el,
ami arra utal, hogy a játékok sikeressége és a bemeneti jellemzők közötti kapcsolat
részben nemlineáris jellegű.
Ugyanakkor a logisztikus regresszió továbbra is stabil és jól értelmezhető baseline modellt
biztosít, amely megfelelő kiindulópontot jelent a további modellek értékeléséhez.

\Section{Domain 2: Multiplayer vs. singleplayer osztályozás}

\subsection{Az alkalmazott modellek áttekintése}

A multiplayer vs. singleplayer osztályozási feladatra több különböző
megközelítést alkalmaztam.
Kiindulásként kizárólag a játékok szöveges leírásaira épülő
baseline modelleket használtam,
majd ezek teljesítményét strukturált metaadatok
(bejegyzett \textit{tagek} és \textit{műfajok}) bevonásával is megvizsgáltam.

Az összehasonlítás során az alábbi modellek kerültek alkalmazásra:
\begin{itemize}
    \item TF--IDF\cite{tf-idf} + logisztikus regresszió\cite{logistic regression} (csak szöveges jellemzők),
    \item TF--IDF + lineáris SVM\cite{svm} (csak szöveges jellemzők),
    \item TF--IDF + logisztikus regresszió strukturált jellemzőkkel,
    \item TF--IDF + lineáris SVM strukturált jellemzőkkel.
\end{itemize}

A modellek egységes tanító--teszt felosztás mellett kerültek kiértékelésre,
stratifikált mintavételezéssel,
annak érdekében, hogy az osztályok torz eloszlása
ne befolyásolja a teljesítménymérést.

\subsection{Teljesítménybeli különbségek elemzése}

Az egyes modellek teljesítményét több metrika mentén értékeltem,
beleértve a pontosság, precízió, recall és F1-mérték\cite{metrics} értékeket.
Az összehasonlítás során különös figyelmet kapott
a kisebbségi multiplayer osztály felismerési képessége,
mivel az adathalmaz jelentősen kiegyensúlyozatlan.

Az eredmények alapján megállapítható,
hogy a lineáris SVM\cite{svm} modellek következetesen jobb teljesítményt nyújtottak,
különösen a recall és az F1-mérték tekintetében.
Ez arra utal, hogy ezek a modellek hatékonyabban kezelik
a nagy dimenziós, ritka TF--IDF\cite{tf-idf} reprezentációkat,
mint a logisztikus regresszió.

A kizárólag szöveges jellemzőkre épülő modellek
már önmagukban is jó teljesítményt értek el,
azonban a strukturált jellemzők bevonása
minden esetben további javulást eredményezett.

A \ref{tab:model_comparison} táblázat egyértelműen mutatja,
hogy a strukturált jellemzők bevonása minden vizsgált modell esetében
javulást eredményezett, különösen a kisebbségi multiplayer osztály
visszahívási arányát és F1-mértékét tekintve.

\begin{table}[H]
\centering
\caption{A különböző modellek teljesítményének összehasonlítása a multiplayer vs. singleplayer osztályozási feladatra}
\label{tab:model_comparison}
\begin{tabular}{lcccc}
\hline
\textbf{Modell} & \textbf{Jellemzők} & \textbf{Accuracy} & \textbf{Recall (MP)} & \textbf{F1 (MP)} \\
\hline
LogReg & TF--IDF (szöveg) & 0.933 & 0.852 & 0.829 \\
LinearSVC & TF--IDF (szöveg) & 0.937 & 0.833 & 0.833 \\
LogReg & Szöveg + tagek + műfajok & 0.955 & 0.893 & 0.883 \\
LinearSVC & Szöveg + tagek + műfajok & \textbf{0.958} & \textbf{0.880} & \textbf{0.889} \\
\hline
\end{tabular}
\end{table}

A legjobb összteljesítményt a TF--IDF szöveges reprezentációt
és a strukturált metaadatokat együttesen alkalmazó
lineáris SVM modell érte el,
ami alátámasztja a korábbi fejezetekben levont következtetéseket.

\subsection{A strukturált jellemzők hatása}

A tagek és műfajok bináris (multi-hot) reprezentációjának hozzáadása
különösen a multiplayer osztály visszahívási arányát növelte.
Ez azt jelzi, hogy a strukturált metaadatok
explicit módon hordozzák a játékmechanikákra vonatkozó információkat,
amelyek hatékonyan kiegészítik
a természetes nyelvű leírásokból kinyerhető mintázatokat.

A kombinált modellek esetében
csökkent a hamis negatív besorolások száma,
ami a kisebbségi osztály szempontjából
kiemelten fontos javulást jelent.

\subsection{Összegző értékelés}

Az összehasonlító elemzés alapján megállapítható,
hogy a TF--IDF\cite{tf-idf} alapú szöveges reprezentációt
és a strukturált metaadatokat együttesen alkalmazó
lineáris SVM\cite{svm} modell bizonyult a leghatékonyabb megoldásnak
a multiplayer vs. singleplayer osztályozási feladatra.

Ez a modell nemcsak magas összesített pontosságot ért el,
hanem kiegyensúlyozott teljesítményt mutatott mindkét osztály esetében,
miközben a döntési mechanizmusa
értelmezhető és szakmailag indokolható maradt.
Ez különösen fontos szempont
elemzési és kutatási célú alkalmazásokban,
valamint a későbbi továbbfejlesztések szempontjából is.
